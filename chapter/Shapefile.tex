% Tugas 2 Kelompok 4
% Akbar Pambudi Utomo (1154094)
% Julham Ramadhana (1154069)
% Andi Wadi Afriandyka (1154113)
% Andi Nurfadillah Ali ()
% Hanna Theresia Siregar ()
% Pebridayanti Hasibuan ()


\section{Shapefile}
\subsection{Pengertian Shapefile}
Shapefile ArcView memiliki format data tersendiri yang disebut dengan shapefiles. 
Shapefiles adalah format data yang menyimpan lokasi geometrik dan informasi atribut dari suatu feature geografis. 
Pada umumnya kita hanya butuh satu file kerja seperti file Microsoft Worl dengan extension file *.doc, 
akan tetapi shapefile memiliki perbedaan, yaitu bahwa satu shapefile memiliki beberapa file yang saling berkaitan satu sama lainnya. 
Beberapa file ini memiliki extension yang 42 berbeda-beda yang disimpan dalam workspace yang sama.
Catatan : tiga file extension pertama adalah bagian file extension yang harus ada dalam sebuah shapefile, file extension berikutnya sifatnya optional.
Fitur geografis di shapefile dapat ditunjukkan oleh titik, garis, atau poligon (area). Ruang kerja yang berisi shapefile
mungkin juga berisi table dBase, yang dapat menyimpan atribut tambahan yang dapat digabungkan ke fitur shapefile.
Semua file yang memiliki ektensi seperti file .txt, .asc, .csv atau tab muncul di ArcCatalog sebagai file text secara default.
Akan tetapi, pada kotak dialog Opsi Kita dapat memilih tipe file mana yang harus direpresentasikan sebagai file
teks dan seharusnya tidak ditampilkan di pohon Catalog. Ketika file teks berisi nilai koma dan tab-delimited,
kita bisa melihat isi file di tampilan table ArcCatalog dan menggabungkannya ke dalam fitur geografis. file teks bisa juga kita hapus, tetapi isinya hanya bisa dibaca di ArcCatalog.
"Shapefile" adalah seperangkat file komputer yang digunakan untuk menyimpan informasi geografis (mis., Batas saluran sensus) 
dan tabel atribut yang terkait dengan informasi geografis (mis., Perumahan sensus dan karakteristik demografis). Shapefiles dapat 
dimanipulasi menggunakan sistem informasi geografis (SIG); ArcView 8.3 (ESRI, Redlands, CA) digunakan dalam proyek ini. 
Paket pajak shapefile diperoleh dari kantor penilai pajak Fulton dan Gwinnett County. Shapefile ini mengandung poligon yang sesuai 
dengan lokasi dan dimensi setiap paket tanah kena pajak di county. Alamat setiap paket disimpan dalam tabel atributnya.

\subsection{Struktur Data Shapefile}
Geodatabase adalah struktur data yang kuat dan canggih. selain topologi yang anda dapatkan secara 
gratis, area dan sekeliling area yang digambarkan dengan fitur linier. Namun, ESRI juga mendukung 
struktur data yang jauh lebih rumit: shapefile. sebuah shapefile dapat menggabungkan dua elemen 
penting yang dimiliki oleh geodatabase(komponen geografis dan database atribut) Perangkat lunak 
basis data adalah sistem manajemen basis data yang dinamai dBASE. Shapefile tertentu dibatasi hanya untuk mewakili satu dari jenis berikut: titik, multipoint, polyLines, atau poligon dengan titik. masing-masing titik memiliki catatan database relasional. Jika sejumlah titik dianggap objek yang sama, maka objek tersebut hanya memiliki satu record di tabel atribut. seperti pada geodatabases, polylines dapat disusun dari satu atau lebih jalur terhubung ataupun terputus-putus. Namun, jalur diperbolehkan untuk disusun hanya dari segmen garis lurus.

\subsection{Daftar beberapa file extension}
*.shp - File yang menyimpan feature geometri (diperlukan dalam sebuah shapefile) 
*.shx - File yang menyimpan index dari feature geometri (diperlukan dalam sebuah shapefile) 
*.dbf - File dBASE yang menyimpan informasi atribut dari suatu feature (diperlukan dalam sebuah shapefile) 
*.sbn dan *.sbx – File yang menyimpan spatial index dari feature (optional) 
*.fbn dan *.fbx – File yang menyimpan spatial index dari feature shapefile yang read-only (optional) 
*.ain dan *.aih – File yang menyimpan index atribut dari field yang aktif dalam sebuah tabel (optional) 
*.prj - File yang menyimpan informasi koordinat dari sebuah shapefile, file ini dapat muncul jika kita menggunakan ArcView Projection Utility (optional).

\subsection{contoh penggunaan kodingan shapefile}
read.shapefile (form.name) read.shp
(shp.name) read.shx (shx.name)
read.dbf (dbf.name, sundulan = FALSE)
write.shapefile (shapefile, out.name, arcgis = FALSE) write.shp (shp,
out.name) write.shx (SHX, out.name)
write.dbf (DBF, out.name, arcgis = FALSE) calc.header
(shapefile) add.xy (shapefile)

bentuk file
scaleXY (shapefile, scale.factor)
convert.to.shapefile (shpTable, attTable, bidang, jenis) convert.to.simple
(shp)
change.id (shpTable, newFieldAsVector) dp (poin,
toleransi)

