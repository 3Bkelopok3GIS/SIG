\section{Perulangan Pada Python}
Perulangan dalam bahasa pemrograman berfungsi untuk memerintahkan komputer melakukan sesuatu secara berulang-ulang. Terdapat dua jenis perulangan dalam bahasa pemrograman python, yaitu perulangan dengan for dan while.
\subsection{While dan For}
Perulangan while disebut dengan uncounted loop (perulangan yang tak terhitung), sementara perulangan for disebut dengan counted loop (perulangan yang terhitung). Perbedaannya adalah ada statement while, biasanya memiliki ciri berupa pengecekan kondisi dan perulangan dilakukan diawal. Sedangkan pada statement for, memiliki ciri berupa inisialisasi perulangan dilakukan diawal statement dan perulangan tersebut akan berhenti ketika syarat/ kondisi yang telah ditentukan terpenuhi\cite{santoso2009bahasa}.

\subsubsection(While Loop)
Perintah While pada pyton biasanya memiliki ciri berupa pengecekan kondisi dan perulangan dilakukan diawal\cite{santoso2009bahasa}. Alur prosesnya adalah ketika sebuah program dijalankan dan kemudian menemukan sebuah kondisi dimana menggunakan loop atau perulangan while, jika kondisi true maka statment itu akan dieksekusi kemudian akan di cek lagi kondisinya. Setelah selesai statmentnya masih true dieksekusi lalu akan mengecek lagi kondisinya dan terus seperti itu, dan jika false statementntya maka akan keluar dari perulangan dan akan melanjutkan ke program selanjutnya.

\subsection{Perulangan (for loop)}
FOR Loop dipakai untuk melakukan perulangan atau iterasi mencapai batas atau jarak yang telah ditentukan/cite{santoso2009bahasa}.

Kegunaan
1.    Ketika ingin pergi ke item urutan tertentu seperti pada list atau string
2.    Ketika ingin melakukan perulangan kode beberapa kali
For interaksi_var in ururan
Statements
Print(“bukan bagian perulangan”) 

\subsection{Perulangan pada python}
\section{For Loop}
Pengulangan for digunakan untuk pengulangan dengan muatan yang banyak.
Contohnya Seperti :
\begin{equation}
for a in range(0, 10):
	print a
\end{equation}
Hasil Outputnya :
\begin{equation}
python for.py
0
1
2
3
4
5
6
7
8
9
\end{equation}

\section{While Loop}
Pengulangan while biasanya digunakan untuk sesuatu yang ga pasti.
Contohnya Seperti :
\begin{equation}
a = 0
while true:
	if a < 10:
		print "saat ini a bernilai: ", a
		a = a + 1
	else a >= 5:
		break
\end{equation}
Hasil Outputnya :
\begin{equation}
python while.py
saat ini a bernilai: 0
saat ini a bernilai: 1
saat ini a bernilai: 2
saat ini a bernilai: 3
\end{equation}

\section{For looping with list}
Contohnya Seperti :
\begin{equation}
hero_dota2_character = ["Mirana", "Axe", "Tusk"]
for character in hero_dota2_character:
	print character
\end{equation}
Hasil Outputnya :
\begin{equation}
python for-list.py
Mirana
Axe
Tusk
\end{equation}
