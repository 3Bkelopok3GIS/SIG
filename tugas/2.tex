tugas
3A
1. sejarah, definisi, identation dan instalasi python
2. jenis2 variable python dan cara penggunaannya
3. perulangan di python dengan contohnya
4. fungsi dan kelas di python dengan contohnya


3B 
1. tentang OGC
2. data Raster
3. data Vektor
4. wmts

3C
1. point
2. line
3. poligon
4. shapefile

3D
1. openlayer
2. leafletjs
3. wms
4. qgis




1. Tetap Menggunakan Format pada tugas sebelumnya (50)
2. di github satu kelompok membuat organization nama kelompoknya dan masukan semua anggota ke grup tersebut (5)
3. fork buku gis https://github.com/BukuInformatika/SIG ke orgz tersebut (5)
4. commit sehari(min 50 kata) per anggota kelompok selama 6 hari (30)




Format Tambahan :
1. untuk list nomor gunakan
	contoh :
	berikut nama anggota kelompok
\begin{enumerate}
	\item darso
	\item karyo
	\item doyok
\end{enumerate}

2. spesial karakter menggunakan tanda \ didepannya
	contoh :
	\&
	\"dalam kurung\"
	jika spesial karakter menjadi banyak atau satu baris gunakan verb
	contoh :
	\verb|%$'%&$&'%'%'%&'%|
	
3. untuk tabel gunakan table , contoh

\begin{table}[h]
\caption{Small Table}
\centering
\begin{tabular}{ccc}
\hline
one&two&three\\
\hline
C&D&E\\
\hline
\end{tabular}
\end{table}

4. untuk rumus gunakan tag equation
	contoh:
	Rumus bola:

	a) Luas permukaan
	 \begin{equation}
	     L = 4 \pi r^2 \,
	\end{equation}
	b) Volume
	 \begin{equation}
	     V = \frac{4}{3}\pi r^3
	\end{equation}