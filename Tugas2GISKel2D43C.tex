\section{Tipe Data Vektor}
\subsection{Pengertian}
Data vektor adalah data yang disimpan dalam bentuk koordinat titik yang menampilkan, 
menempatkan, dan menyimpan data spacial dengan menggunakan titik, garis atau polygon.
Terdapat tiga jenis tipe data vektor yaitu titik, garis, dan polygon. Tipe data ini 
biasanya terdapat pada peta. Setiap bagian dari data vektor bisa saja mempunyai 
informasi yang berasosiasi satu sama lain.

\section{Data Vektor Line}
\subsection{Pengertian}
Line merupakan bahasa Inggris dari garis. Garis adalah bentuk geometri liniar yang
 menghubungkan dua titik atau lebih dan biasanya digunakan untuk mempresentasikan
 object berdimensi satu. batas object geometri polygon juga merupakan sebuah garis-garis,
 begitu pula dengan jaringan listrik, jaringan komunikasi, jaringan air minum, saluran buangan,
 dan utility lain yang dapat dipresentasikan sebagai object dengan bentuk geometri garis.
 Hal itu pula yang akan bergantung pada skala peta yang menjadi sumbernya atau skala
 representasi akhirnya.
 Garis bisa digunakan untuk menunjukkan route suatu perjalanan atau menggambarkan boundary. 
 Poligon bisa digunakan untuk menggambarkan sebuah danau atau sebuah Negara pada peta dunia. 
 Dalam format vektor, bumi direpresentasikan sebagai suatu mosaik dari garis (arc/line), 
 poligon (daerah yang dibatasi oleh garis yang berawal dan berakhir pada titik yang sama), 
 titik/ point (node yang mempunyai label), dan nodes (merupakan titik perpotongan antara dua baris).
