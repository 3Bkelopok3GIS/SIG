%leafletjs
%kelompok 2 D4 TI-2D
%Cahya Kurniawan 1154038      
%Doni Saputra 11540030   
%Ika Syam Setiawati 1154053     
%Silvy Dharma Febryana 1154112
%Widi damayanti 1154062
 
 
 
 \section{sejarah leafletjs}
        Leaflet adalah JavaScript Library terkemuka yang berifat opensource untuk membangun peta interaktif yang Mobile friendly. Dengan ukuran hanya sekitar 38 KB, ia memiliki semua fitur pemetaan yang dibutuhkan sebagian besar pengembang.
    Kelebihannya karena opensource lebih mudah dikembangkan oleh peneliti selanjutnya dan mudah untuk mengadaptasi teknologi baru pada GIS. Pada penerapannya SIG memerlukan data spasial yaitu data yang merujuk kepada posisi sebuah objek dalam bentuk koordinat dalam ruang bumi. GIS adalah sistem yang dirancang untuk memperoleh, menyimpan, mengupdate, memanipulasi, menganalisis dan menampilkan semua bentuk informasi yang berefensi geografis.
Dengan penggunaan leaflet, data-data spasial seperti gedung dan ruangan yang berupa format geoJson dapat disimpan didalam server, tanpa harus terhubung ke internet hanya dengan menggunakan intranet.Untuk mengakses data-data tersebut digunakan plugin jQuery dan bootstrap untuk menampilkan peta ke halaman browser. Kelebihan menggunakan leaflet adalah leaflet menyediakan fungsionalitas untuk menambahkan penanda, pop up, garis overlay, dan bentuk menggunakan lapisan, zoom, pan, tapi ini hanya fitur ini leaflet.

\section{penggunaan leafletjs}
Leaflet merupakan alternative baru bagi para perintis peta web, seperti open layers ataupun google maps api. Ini juga dapat meringankan open source dan bertujuan untuk membentuk dan membantu mengembangkan dalam proses pembuatan peta yang indah yang compatible di seluruh pc (desktop) dan juga ponsel tanpa harus mengorbankan performa dari apa yang terjadi ketika selesai pembuatan


