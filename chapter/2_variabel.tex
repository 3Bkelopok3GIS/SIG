\section{Variabel}
Pada sebagian besar bahasa pemrograman, nama suatu variabel
menjelaskan suatu nilai dengan tipe data tertentu 
dan menempati alamat memory yang pasti.
Variabel menyimpan data atau nilai yang dilakukan selama program dieksekusi,
Nilai variabel tersebut dapat diganti-ganti, namun tipe data selalu tetap.
Tidak demikian dengan python dimana tipe datanya dapat diubah-ubah
secara dinamis\cite{suparno2013komputasi}.

Variabel adalah entitas yang memiliki nilai dan berbeda satu dengan yang lain. Variabel mengalokasikan memori untuk menyimpan nilai.
Hal ini berarti ketika anda membuat variabel maka anda memesan beberapa ruang di memori. 
Variabel bisa digunakan untuk menyimpan bilangan bulat, desimal juga karakter.
Python sangat mementingkan indentasi, sehingga kita perlu melakukan indentasi secara konsisten. 
Indentasi tersebut dipermudah dengan penggunaan tombol Tab dan dimulai dari kolom pertama untuk setiap blok baru. \cite{santoso2009bahasa}


Variabel pada Python memiliki beberapa aturan seperti :
•	Case Sensitive : penggunaan huruf besar dan huruf kecil yang dibedakan.
•	Harus dimulai dengan underscore (_) atau huruf biasa, setelah itu dapat diikuti dengan huruf, angka atau underscore (_).
•	Tidak boleh mengandung karakter special seperti !,@,\#,\$ dan lainnya.
•	Hanya dapat menggunakan suatu variable setelah kita memberikan nilai ke dalamnya atau telah dilakukan assignment.
•	Setiap variable akan menyimpan referensi ke suatu objek dalam memory.\cite{santoso2009bahasa}


Variabel adalah sebuah nama yang selalu menunjukkan nilai tertentu. Dalam bahasa pemrograman Python, untuk membentuk sebuah variabel itu cukup hanya memberi nama pada nilai yang akan dibuat. Ini disebut dengan Assignment.
Contoh : 
>>>pesan="Halo, Semuanya"
Contoh di atas membuktikan bahwa itu adalah sebuah assignment, yang memberi kan sebuah nilai "Halo, Semuanya".\cite{Utami2004logika}

contoh lainnya:
\begin{equation}
>>> b = 2 # b bilangan bertipe integer
>>> print b
2
>>> b = b * 2.0 # Sekarang b bilangan bertipe float
>>> print b
4.0
\end{equation}
Tulisan b = 2 artinya kita memberi nilai pada variabel b dengan angka 2 yang bertipe integer
(bilangan bulat). Statemen berikutnya adalah melakukan operasi perkalian b ∗ 2.0 lalu hasilnya
disimpan pada variabel yang sama yaitu variabel b. Dengan demikian nilai b yang lama
akan diganti dengan nilai b yang baru, yaitu yang berasal dari operasi yang terakhir. 

Python mementingkan indentasi, sehingga perlu indentasi yang konsisten. Indentasi dipermudah sesuai penggunaan
tombol Tab dan dimulai dari kolom pertama untu setiap blok baru. 

Dalam menulis program, akan menggunakan code yang pernah kita  buat atau ditulis sebelumnya, pasti
kita gunakan kembali, dengan beberapa nilai berbeda.
 
Tentu saja tidak mungkin menuliskan kembali kode yang ingin dipanggil ulang tersebut.
Solusinya, supaya dapat dikelompokan kode-kode yang sering dipanggil ulang dalam suatu kelompok.

Selain itu dapat memecah masalah-masalah yang besar  menjadi masalah-masalah yang lebih kecil.
Dalam C atau bahasa pemrograman lain, biasanya digunakan istilah function.

Kemampuan python dalam mengelola tipe data sangat baik. Untuk mendeklarasikan suatu variabel dilakukan secara langsung tanpa menyebutkan tipe datanya, ini yang membedakan python dengan bahasa lain. Python akan menentukan tipe datanya secara otomatis. Python juga mendukung konversi dan perhitungan antar tipe data dengan ketelitian yang tinggi. Python membagi tipe data ke dalam 2 jenis bilangan (semua tipe yang berhubungan dengan angka murni) dan string. Untuk tipe data dalam rumpun bilangan termasuk didalamnya adalah integer, long, float, oktal, hexadimal dan bilangan kompleks. Hal-hal yang harus diperhatikan :
•	Untuk bilangan oktal dan hexa masing-masing diawali dengan 0 dan 0x
•	Untuk bilangan yang panjang diakhiri dengan karakter l atau L
•	Untuk bilangan float, gunakan e atau E pada eksponensial
•	Untuk bilangan kompleks dibagi ke menjadi bagian real dan imajiner, dan diakhiri dengan j atau J Operator untuk tipe dalam rumpun bilangan. \cite{utamipemrograman}
