\section{Perulangan Pada Python}
Perulangan dalam bahasa pemrograman berfungsi untuk memerintahkan komputer melakukan sesuatu secara berulang-ulang. Terdapat dua jenis perulangan dalam bahasa pemrograman python, yaitu perulangan dengan for dan while.
\subsection{While dan For}
Perulangan while disebut dengan uncounted loop (perulangan yang tak terhitung), sementara perulangan for disebut dengan counted loop (perulangan yang terhitung). Perbedaannya adalah ada statement while, biasanya memiliki ciri berupa pengecekan kondisi dan perulangan dilakukan diawal. Sedangkan pada statement for, memiliki ciri berupa inisialisasi perulangan dilakukan diawal statement dan perulangan tersebut akan berhenti ketika syarat/ kondisi yang telah ditentukan terpenuhi\cite{santoso2009bahasa}.

\subsubsection(While Loop)
Perintah While pada pyton biasanya memiliki ciri berupa pengecekan kondisi dan perulangan dilakukan diawal\cite{santoso2009bahasa}. Alur prosesnya adalah ketika sebuah program dijalankan dan kemudian menemukan sebuah kondisi dimana menggunakan loop atau perulangan while, jika kondisi true maka statment itu akan dieksekusi kemudian akan di cek lagi kondisinya. Setelah selesai statmentnya masih true dieksekusi lalu akan mengecek lagi kondisinya dan terus seperti itu, dan jika false statementntya maka akan keluar dari perulangan dan akan melanjutkan ke program selanjutnya.
