%kelompok 3 kelas 3B
%Diana Satima Gistivani 1154018



\section{data vektor}
  Data vektor dalam Sistem Informasi Geografis. dalam  data vektor bumi direpresentasikan sebegai suatu  mosaic yang terdiri dari garis (arclline), polygon (dareah yang dibatasi oleh garis yang berawal dan berakhir pada titik yang sama), titik/point (node yang memiliki label),  dan nodes (titik perpotongan antara dua buah garis). Model data vektor sendiri merupakan model yang banyak digunakan, model ini berbasis pada titik (points) dengan nilai koordinat (x,y) untuk membangun objek spasialnya. 

\subsection {Objek pada datavektor}
Objek yang dibangun terbagi menjadi tiga bagian lagi, yaitu berupa titik (point), garis (line), dan area (polygon). 
a. titik (point) merupakan representasi grafis yang paling sederhana pada suatu objek. Titik tidak mempunyai dimensi tetapi dapat ditampilkan dalam bentuk symbol baik pada peta maupun dalam layar monitor.
b. garis (line) merupakan bentuk linear yang menghubungkan dua ataui lebih titik dan merepresentasikan objek dalam satu dimensi. 
c. Area (polygon) merupakan representasi objek dalam dua dimensi.
