%WMS
%Kelompok 3 D4 TI-3D
%Aditya Pratama Dharma-1154043
%Andi Syahjaratu Daur-1154092
%Bendra Wardana-1154015
%Dini Islamiani Lestari-1154039
%Nur Rahmawati-115124

/section{deskripsi wms}
  Web Map Service adalah salah satu jenis penggambarakn OGC Layanan model layanan web dan ia menyedian platfrom
multi-interoperability.Karya ini menghadirkan sebuah metode untuk menerapkan layanan peta web OGC berdasarkan teknik
Layanan Wen dan memperkenalkan proses terperinci.
  
    Web Map Service (wms) memberikan informasi kepada pengguna internet oleh tata ruang peta gambar.Umumnya,yang tersimpan
di dalam tata ruang data tersebut data vektor itu adalah panjang untuk menciptakan data vektor peta gambar.
setiap sub-rectangle dikirim ke sebuah wmssub-maps node untuk menghasilkan sekumpulan peta yang dihasilkan merekontruksi
dengan semuanya sub-maps.Semua sub-maps dihasilkan di paralel,jadi makin sedikit waktu seluruh yang habis memproduksi
sebuah peta.

   Web Map Service (WMS) memberikan informasi kepada pengguna internet oleh tata ruang peta gambar.Umumnya,
yang tersimpan di dalam tata ruang data tersebut data vektor .Itu adalah panjang ayub untuk 
menciptakan data vektor peta gambar dari .Biaya untuk mengurangi waktu , maka kami memanfaatkan linux cluster .
Ketika peta diminta, itu adalah suatu koordinat lingkup didefinisikan dengan xmin persegi panjang,ymin,xmax, 
ymax harus dispesifikasikan .Kami merancang beban keseimbangan menurut algoritma untuk membagi permintaan ke dalam 
beberapa sub-rectangles .

   Setiap sub-rectangle dikirim ke sebuah wms sub-maps node untuk menghasilkan sekumpulan .
Peta yang dihasilkan akan merekonstruksi dengan semuanya ini sub-maps .Semua ini sub-maps dihasilkan di paralel, 
jadi makin sedikit waktu yang seluruh habis di memproduksi sebuah peta .Bagaimana untuk membagi menurut adalah 
kunci masalah permintaan .Pertama , kami menghadirkan metode untuk menghitung tingkat 2d bobot beban distribusi peta lingkup .
Kedua , node beban kemampuan yang dievaluasi .Kemudian , penulis hadirmetode menurut untuk membelah seluruh.
Kertas ini juga membahas algoritma kinerja pelaksanaan .

   Pengenalan Peta map Service (WM)s menghasilkan peta secara spasial direferensikan data dari informasi geografis secara dinamis. 
Standar internasional ini mendefinisikan sebuah "peta" untuk menjadi penggambaran tentang informasi geografis 
sebagai file gambar digital cocok untuk ditampilkan pada layar komputer. Peta tidak data itu sendiri. 
Wms-dihasilkan maps umumnya diterjemahkan dalam format bergambar seperti PNG,GIF atau JPEG,atau kadang-kadang sebagai
elemen-elemen grafis berbasis vektor dalam Scalable Vector Grafis (SVG) atau Komputer Web Metafile Grafis (WebCGM format).
