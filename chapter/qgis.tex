%qgis
%kelompok 4 D4 TI-2D
%Ayu Permata Sari        1154022
%Librantara Erlangga     1154071
%Martin Luter Zega       1154120
%Putri Aulia Ramadhanie  1154096
%Ryan Hafizh Herdiana    1154067
%Copyright (c) 2017 Copyright Holder All Rights Reserved.

\section{pengenlan qgis}
	  qgis(dulu dikenal dengan nama quantum GIS) adalah salah satu perangkat lunak open source untuk keperulan GIS(Geographic information systems) yang paling populer, dengan basis penggunanya yang terus meningkat dan berkembang.
	 aplikasi ini dapat melakukan banyak hal yaitu pembuatan data spasial,editing, dan juga analisis pemetaan.Selain berbasis dekstop aplikasi ini juga menyediakan server digunakan dalam web dan bahkan qgis siap digunakan dalam hp.

	Quantum GIS dikembangkan dengan bahasa pemrograman C++ dan bersifat multi platform. Salah satu perangkat lunak ini dapat dijalankan pada berbagai sistem operasi. Saat ini, versi binary (installer) Quantum GIS tersedia untuk sistem operasi Microsoft Windows, Linux (berbagai varian distro), FreeBSD dan MacOS X. Quantum GIS bahkan sudah mulai dicoba dijalankan di sistem operasi Android yang banyak digunakan di perangkat mobile (smartphone/tablet). Saat ini versi stabil Quantum GIS adalah 1.8.0, dan sedang dalam tahap pengembangan untuk mencapai versi mayor 2.0.

	Quantum GIS dapat digunakan untuk pengolahan data atribut secara umum seperti melakukan overlay layer, menguhitung luasan suatu wilayah, memberikan informasi tambahan pada suatu titik, ataupun merancang layout peta. Quantum GIS juga mendukung penggunaan GPS sehingga pengguna dapat langsung memindahkan data langsung dari GPS ke PC atau sebaliknya. Quantum GIS sangat layak untuk dijadikan alternative perangkat lunak pemetaan dalam berbagai keperluan.

	Quantum GIS terdiri dari beberapa modul, yaitu :
	\begin{enumerate}
		\item QGIS Desktop : untuk input data, penampilan data, query data, analisa data, dan presentasi dalam bentuk peta.
		\item QGIS Browser : untuk manajemen data (copy, paste, preview, delete).
		\item QGIS Server : untuk mengelola basis data special dan layanan peta yang mendukung koneksi web site.
		\item QGIS Client : framework aplikasi GIS berbasis web yang mengkonsumsi data dari Quantum GIS Server.
	\end{enumerate}

	 \subsection{latar belakang}
		qgis merupakan perangkat lunak berbasis open source yang merupakan projek dari GIS,yang dimulai pada tahun 2002. Tujuan awal proyek ini adalah untuk membuat spasial data. hari ini qgis sudah mencapai titik dimana qgis berevolusi di mana qgis digunakan oleh berbagai pengguna setiap hari untuk melihat data spasial,mengedit juga menganalisis , terutama dalam kegiatan pembelajaran.
	 qgis banyak digunakan oleh orang yang menggunakan platform linux dan unix , disusul oleh windows dan mac os x dan telah di publikasikan dibawah lisensi GNU(kumpulan software gratis).core qgis dikembangkan dengan toolkit qt dan c++.selain itu qgis juga menyediakan api(application program interface)python, yang diunakan untuk memperluas jangakuan fungsinya.Sebuah api Python yang powerful dapat membantu memanfaatkan kemampuan secara efisien GIS dan mengintegrasikan berbagai alat dan bahasa pemrograman untuk mengembangkannya.
	 effeknya, semakin meningkatan jumlah kontribusi yang telah sangat terlihat sejak diperkenalkannya API Python QGIS di QGIS 0.9 pada tahun 2007 ketika pengembang baru mulai menambahkan fungsionalitas menggunakan plugin Python.

	 \subsection{pengembagan qgis}
		Melakukan proses penulisan ulang dari proyek Sistema Extremeño de Análisis Territorial (SEXTANTE),dimana proyek ini diluncurkan pada tahun 2004. awalnya, SEXTANTE ditulis dengan menggunakan bahasa java dan berjalan pada GIS desktop.namun akhirnya, SEXTANTE menjadi sebuah library terpisah dengan framework analisis,seperangkat algoritma, alat grafis untuk menggunakan algortimanya dan menjelaskannya melalu analisis workflow.
	 Sebenarnya algoritma yang ada pada SEXTANTE pada versi java merupakan adaptasi dari proyek System for Automated Geoscientific Analyses (SAGA),yang dimana merupakan GIS berbasis desktop dengan analisi tingkat lanjut.Keputusan untuk mengubah SEXTANTE menjadi library terpisah memungkinkannya menggabungkannya ke GIS berbasis Java lainnya, seperti OpenJUMP, uDig, Kosmo dan OrbisGIS.
	 Kemajuan paling pesat dalam kerangka desain diperlihatkan pada saat migrasi ke qgis ketika kerangka/inti SEXTANTE di buat ulang sebagai plugin pada pyhton.bukan melakukan porting algortima ke dalam python melainkan pendekatan baru yang dilakukan yaitu membiarkan SEXTANTE untuk terhubung ke dalam aplikasi. Dengan demikian, binari dari SAGA yang asli dapa meningkatan kemampuan analisis.
	 Pada tahun 2012, SEXTANTE digunakan sebagai plugin inti pada qgis dan diganti namanya menjadi processing\cite{graser2015processing}.Pada saat ini kerangka pemrosesaan data geospasial pada qgis desktop gis juga pada versi javanya telah terintergrasi dengan SEXTANTE dalam gvSIG, dan pengembangannya dilakukan dalam bagian gvSIG. dan oleh karena itu SEXTANTE framework tidak tersedia lagi sebagai lbrary yang berdiri sendiri.

\section{Fungsi-Fungsi}
	\subsection{melihat data}
	Anda dapat melihat dan overlay data vektor dan raster dalam format dan proyeksi yang berbeda tanpa konversi ke format internal maupun umum. Format yang didukung termasuk:
	Tabel spasial-enabled dan tampilan menggunakan PostGIS, SpatiaLite dan MSSQL Spasial, Oracle Spasial, format vektor yang didukung oleh perpustakaan OGR, termasuk ESRI shapefile, MapInfo, SDTS, GML dan banyak lagi, lihat bagian Pekerjaan dengan Data Vektor.
	Format raster dan citra yang didukung dengan terpasangnya GDAL (Geospatial Data Abstraction Library) perpustakaan, seperti GeoTiff, ERDAS IMG, ArcInfo ASCII GRID, JPEG, PNG dan banyak lagi, lihat bagian Pekerjaan dengan Data Raster.
	Data raster dan vektor GRASS dari basis data GRASS (lokasi/mapset). Lihat bagian GRASS GIS Integration.
	Data spasial dalam jaringan sebagai Layanan OGC Web, termasuk WMS, WMTS, WCS, WFS, dan WFS-T. Lihat bagian Pekerjaan dengan Data OGC.
	\subsection{Explore data and compose maps}
	Anda bisa menyusun peta dan secara interaktif mengeksplorasi data spasial dengan GUI yang ramah. Banyak alat bantu yang tersedia di GUI meliputi: QGIS browser, On-the-fly reprojection, DB Manager, Map composer, Overview panel, Spatial bookmarks, Annotation tools, Identify/select features, Edit/view/search attributes, Data-defined feature labelling, Data-defined vector and raster symbology tools, Atlas map composition with graticule layers, North arrow scale bar and copyright label for maps, Support for saving and restoring projects.
	\subsection{Create, edit, manage and export data}
	Anda dapat membuat, mengedit, mengelola dan mengekspor lapisan vektor dan raster dalam beberapa format. QGIS menawarkan yang berikut ini: Alat digitalisasi untuk format yang didukung oleh OGR dan lapisan vektor GRASS, Kemampuan untuk membuat dan mengedit shapefiles dan layer vektor GRASS, Plugin Georeferencer ke gambar geocode, Alat GPS untuk mengimpor dan mengekspor format GPX, dan mengonversi format GPS lainnya ke GPX atau turun / mengunggah langsung ke unit GPS (Di Linux, usb: telah ditambahkan ke daftar perangkat GPS.), Dukungan untuk memvisualisasikan dan mengedit data OpenStreetMap, Kemampuan untuk membuat tabel database spasial dari shapefiles dengan plugin DB Manager, Perbaikan penanganan tabel database spasial, Alat untuk mengelola tabel atribut vektor, Opsi untuk menyimpan tangkapan layar sebagai gambar georeferensi, Alat DXF-Export dengan kemampuan yang disempurnakan untuk mengekspor gaya dan plugin untuk melakukan fungsi seperti CAD.
	\subsection{Analyze Data}
	Anda bisa melakukan analisis data spasial pada database spasial dan format pendukung OGR lainnya. QGIS saat ini menawarkan analisis vektor, sampling, geoprocessing, geometri dan alat manajemen basis data. Anda juga dapat menggunakan alat GRASS terpadu, yang mencakup fungsionalitas GRASS lengkap lebih dari 400 modul. Atau, Anda dapat bekerja dengan Plugin Pemrosesan, yang menyediakan kerangka analisis geospasial yang kuat untuk memanggil algoritma asli dan pihak ketiga dari QGIS, seperti GDAL, SAGA, GRASS, fTools dan lainnya.
	\subsection{skala}
	Skala adalah perbandingan antara jarak pada model (peta) dan jarak sebenarnya dilapangan. Mengatur skala adalah salah satu cara yang dapat digunakan mengatur zoom karena skala dapat berperan sebagai zoom level. Skala pada peta tersebut ditampilkan pada toolbar dibagian bawah jendela utama QGIS.

	\subsection{tools pada qgis}
	Pengenalan tools:
	\begin{enumerate}
		\item Menu Bar, menyediakan berbagai macam fitur/fungsi pada Quantum GIS yang akan digunakan untuk pengoperasian lebih lanjut. Sebagaian besar fungsi yang ada pada menu bar juga terdapat pada tool bar.
		\item Tool bar, terdiri dari deretan gambar(icon) yang menyediakan akses cepat (shortcut_ pada fungsi-fungsi yang sebagaian besar sama seperti yang ada pada menu bar.
		\item Map legend, menampilkan daftar layer yang sedang dibuka pada project, baik layer yang aktif maupun tidak aktif akan ditampilkan semua. Dapat juga diatur urutan kenampakan layer dengan cara drag dan drop pada layer.
		\item Map view, merupakan area dimana layer-layer ditampilkan. Tampilan pada map view tergantung dari layer yang aktif
		\item status bar, menunjukkan posisi koordinar cursor berada, dengan catatan yang ada pada jendela map view sudah memiliki Georeference. Selain menunjukkan posisi, status bar juga menunjukkan skala tampilan pada mapview.
	\end{enumerate}

	Tool bar pada baris paling atas qgis :
	\begin{enumerate}
		\item file : toolbar ini berisi alat untuk membuat, membuka, menyimpan, dan mencetak proyek
		\item manage layers : toolbar ini berisi alat untuk menambahkan lapisan dari file vektor atau raster, database, layanan web, file teks, atau membuat lapisan baru.
		\item Database : Saat ini, toolbar ini hanya berisi pengelola DB, tapi database lainnya akan muncul di sini saat dipasang
		\item help : download manual pengguna
	\end{enumerate}

	baris kedua toolbar berisi berikut ini:
	\begin{enumerate}
		\item map navigation : toolbar ini berisi alat pan dan zoom
		\item atribut: alat ini digunakan untuk mengidentifikasi, memilih, membuka tabel atribut, ukuran, dan sebagainya
		\item label: alat ini digunakan untuk menambahkan, mengkonfigurasi, dan memodifikasi label
		\item vector: saat ini kosong, namun akan diisi oleh plugin python tambahan
		\item web: saat ini kosong, tapi akan diisi oleh plugin python tambahan
	\end{enumerate}
