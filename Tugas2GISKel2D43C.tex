\section{Geospasial}
\subsection{Pengertian}
Geospasial terdiri dari dua kata, yaitu geo dan spasial. Geo berarti bumi sedangkan spasial berarti 
ruang. UU No 4 tahun 2011 tentang geospasial menyebutkan, spasial adalah aspek keruangan dari suatu objek, 
atau yang mencakup lokasi, letak, dan posisinya
Data geospasial dipecah menjadi dua, yaitu yang pertama; Data grafis atau geometri. Data ini terdiri dari
tiga elemen : titik, garis, dan luasan. data data ini berbentuk dalam vektor maupun raster. yang kedua
adalah data attribut atau data tematik.

\section{Tipe Data Vektor}
\subsection{Pengertian}
Data vektor adalah data yang disimpan dalam bentuk koordinat titik yang menampilkan, 
menempatkan, dan menyimpan data spacial dengan menggunakan titik, garis atau polygon.
Terdapat tiga jenis tipe data vektor yaitu titik, garis, dan polygon. Tipe data ini 
biasanya terdapat pada peta. Setiap bagian dari data vektor bisa saja mempunyai 
informasi yang berasosiasi satu sama lain.

\section{Data Vektor Line}
\subsection{Pengertian}
Line merupakan bahasa Inggris dari garis. Garis adalah bentuk geometri liniar yang
 menghubungkan dua titik atau lebih dan biasanya digunakan untuk mempresentasikan
 object berdimensi satu. batas object geometri polygon juga merupakan sebuah garis-garis,
 begitu pula dengan jaringan listrik, jaringan komunikasi, jaringan air minum, saluran buangan,
 dan utility lain yang dapat dipresentasikan sebagai object dengan bentuk geometri garis.
 Hal itu pula yang akan bergantung pada skala peta yang menjadi sumbernya atau skala
 representasi akhirnya.
 Garis bisa digunakan untuk menunjukkan route suatu perjalanan atau menggambarkan boundary. 
 Poligon bisa digunakan untuk menggambarkan sebuah danau atau sebuah Negara pada peta dunia. 
 Dalam format vektor, bumi direpresentasikan sebagai suatu mosaik dari garis (arc/line), 
 poligon (daerah yang dibatasi oleh garis yang berawal dan berakhir pada titik yang sama), 
 titik/ point (node yang mempunyai label), dan nodes (merupakan titik perpotongan antara dua baris).
 \section{Raster}
 \subsection{Pengertian}
Data raster (juga dikenal sebagai data grid) mewakili tipe keempat dari fitur: permukaan. 
Data raster berbasis sel dan kategori data ini juga mencakup citra udara dan satelit. 
Ada dua jenis data raster: kontinu dan diskrit. Contoh data raster diskrit adalah kepadatan penduduk. 
Contoh data kontinyu adalah pengukuran suhu dan elevasi. Ada juga tiga jenis dataset raster: data tematik, 
data spektral, dan gambar.Raster dataset adalah intrinsik untuk analisis spasial yang paling. 
Analisis data seperti ekstraksi kemiringan dan aspek dari Digital Elevation Models terjadi dengan dataset raster.
Pemodelan hidrologi spasial seperti ekstraksi daerah aliran sungai dan jalur aliran juga menggunakan sistem berbasis raster.
Data spektral menyajikan citra udara atau satelit yang kemudian sering digunakan untuk memperoleh informasi geologi vegetasi dengan mengklasifikasikan tanda tangan spektral dari setiap jenis fitur.
