TUGAS KELOMPOK : 
BUatlah artikel tentang definisi/sejarah/versi/contoh dari :
Kelas 3B
1. Sejarah ptolemy
2. al idrisi
3.dkk
4. dkk

Kelas 3C
1. Definisi
2. kordinat Lat Long
3. Kordinat LU LS BT BB
4. Sejarah Bumi
5-6 orang

Kelas 3A
1. Diagram Cartesius
2. Bangun Ruang -> Bola
3. Sejarah Benua dan kordinat
4. Sejarah garis khatulistiwa dan prime meridian

Kelas 3D
1. Sejarah, penjelasan, dkk penentuan standar waktu
2. Sejarah atau deskripsi antartika kutub selatan
3. sejarah atau deskripsi kutub utara
4. Penanggalan, pergantian tahun, bulan.


Artikel diketik menggunakan text editor
Notepad ++
Sublime
Atom
vi / vim

1. file disimpan dalam format namatugas.tex

2. gambar disimpan dalam folder figures dengan namagambar

3. referensi wajib dari google scholar,scholar.google.com

4. Setiap referensi yang diambil, maka tuliskan ke dalam 
	file bernama reference.bib
   yang berisi kumpulan bibTex dari referensi nomor 3

5. Gunakan standar pengutipan yang baik dan benar

6. Gambar disebutkan di dalam artikel dengan format \ref{namagambar}
   dan gambar diselipkan dengan menambahkan blok sintaks :
	\begin{figure}[ht]
	\centerline{\includegraphics[width=1\textwidth]{figures/namagambar.JPG}}
	\caption{penjelasan keterangan gambar.}
	\label{namagambar}
	\end{figure}
	Contoh :
	Pada gambar \ref{namagambar} dijelaskan bahwa sistem operasi memiliki 
	3 versi.
	
7. Referensi disebutkan dengan menyebutkan nama di dalam file bibtex No.4
   contoh :
	Jika Bibtex :
	@inproceedings{ganapathi2006windows,
	  title={Windows XP Kernel Crash Analysis.},
	  author={Ganapathi, Archana and Ganapathi, Viji and Patterson, David A},
	  booktitle={LISA},
	  volume={6},
	  pages={49--159},
	  year={2006}
	}
	Maka artikel :
	Dalam sebuah artikel dari Ganapathi yang menyebutkan bahwa komputasi 
	adalah keniscayan \cite{ganapathi2006windows}.
	
	
8. Penyebutan subbab dan subsubbab diatur dengan cara 
	judul sub bab \section{nama sub bab}
	judul sub sub bab ditulis dengan \subsection{judul sub sub bab}
	judul sub sub sub bab ditulis dengan \subsubsection{Judul sub sub sub bab}
	contoh :
	\section{Sejarah Peta}
	Perkembangan peta dunia tidak luput dari para ahli geografi dan kartografi. Peta dunia yang populer pada saat ini merupkan kontribusi dari para 
	pembuat peta sebelumnya

	\subsection{Ptolemy's}
	Ptolemy's diduga membuat peta pada abad ke 2
	
9. Satu kelompok minimal 1500 kata, waktu maksimal 6*24 jam

10. Harus di scan plagiat setiap kelompok dari portal kampus keren, minimal uniqe 80%

