\section{Definisi Data Spasial (GEOGRAPHICS INFORMATION SYSTEM)}
Data Spasial Sistem Informasi Geografis (SIG) model data yang akan digunakan dari bentuk dunia nyata harus dapat diimplementasikan ke dalam basisdata. Data ini dimasukkan ke dalam komputer yang nantinya memanipulasi objek dasar yang memiliki atribut geometri (entity spasial/entity geografis) 
(Prahasta, 2002a). Data spasial pada dasarnya dapat disimpulkan bahwa data spasial merupakan suatu entitas data dalam Sistem Informasi Geografis (SIG) yang dapat dikelola, dianalisa dan dapat memetakan informasi objek keruangan beserta data-data atributnya serta dapat disimpan di dalam database yang dapat ditampilkan kedalam suatu sistem tertentu sehingga dapat mendukung dalam pengambilan keputusan. 

\subsection{Model data Vektor pada Geographics Information System GIS}
Vektor  pada GIS mampu melakukan penempatan, menampilkan data spasial bahkan menyimpan datanya yang menggunakan titik-titik, garis-garis dan juga poligon yang dilengkapi dengan artibut-artibutnya. Bentuk-bentuk dasar representasi dari data spasial ini di dalam sistem model data vektor dapat didefinisikan oleh sistem koordinat kartesian dua dimensi (X,Y). Dimana di dalam model data spasial vektor,
