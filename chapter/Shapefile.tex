% Tugas 2 Kelompok 4
% Akbar Pambudi Utomo (1154094)
% Julham Ramadhana (1154069)
% Andi Wadi Afriandyka (1154113)
% Andi Nurfadillah Ali ()
% Hanna Theresia Siregar ()
% Pebridayanti Hasibuan (1154118)


\section{Shapefile}
\subsection{Pengertian Shapefile}
Shapefile ArcView memiliki format data tersendiri yang disebut dengan shapefiles. 
Shapefiles adalah format data yang menyimpan lokasi geometrik dan informasi atribut dari suatu feature geografis. 
Pada umumnya kita hanya butuh satu file kerja seperti file Microsoft Worl dengan extension file *.doc, 
akan tetapi shapefile memiliki perbedaan, yaitu bahwa satu shapefile memiliki beberapa file yang saling berkaitan satu sama lainnya. 
Beberapa file ini memiliki extension yang 42 berbeda-beda yang disimpan dalam workspace yang sama.
Catatan : tiga file extension pertama adalah bagian file extension yang harus ada dalam sebuah shapefile, file extension berikutnya sifatnya optional.
Fitur geografis di shapefile dapat ditunjukkan oleh titik, garis, atau poligon (area). Ruang kerja yang berisi shapefile
mungkin juga berisi table dBase, yang dapat menyimpan atribut tambahan yang dapat digabungkan ke fitur shapefile.
Semua file yang memiliki ektensi seperti file .txt, .asc, .csv atau tab muncul di ArcCatalog sebagai file text secara default.
Akan tetapi, pada kotak dialog Opsi Kita dapat memilih tipe file mana yang harus direpresentasikan sebagai file
teks dan seharusnya tidak ditampilkan di pohon Catalog. Ketika file teks berisi nilai koma dan tab-delimited,
kita bisa melihat isi file di tampilan table ArcCatalog dan menggabungkannya ke dalam fitur geografis. file teks bisa juga kita hapus, tetapi isinya hanya bisa dibaca di ArcCatalog.
"Shapefile" adalah seperangkat file komputer yang digunakan untuk menyimpan informasi geografis (mis., Batas saluran sensus) 
dan tabel atribut yang terkait dengan informasi geografis (mis., Perumahan sensus dan karakteristik demografis). Shapefiles dapat 
dimanipulasi menggunakan sistem informasi geografis (SIG); ArcView 8.3 (ESRI, Redlands, CA) digunakan dalam proyek ini. 
Paket pajak shapefile diperoleh dari kantor penilai pajak Fulton dan Gwinnett County. Shapefile ini mengandung poligon yang sesuai 
dengan lokasi dan dimensi setiap paket tanah kena pajak di county. Alamat setiap paket disimpan dalam tabel atributnya.
Shapefile ESRI atau biasa disebut shapefila adalah format data geospasial yang umum untuk perangkat lunak sistem informasi geografis, dengan pengertian bahwa shape merupakan properti intrinsik utama untuk sistem visual manusia. manusia lebih sering mengasosialisasikan objek dengan bentuknya ketimbang elemen lainnya (warna misalnya), pada umumnya, citra yang dibentuk oleh mata merupakan citra dwimatra (2 dimensi), sedangkan objek yang dilihat umumnya berbentuk trimatra (3 dimensi). informasi bentuk objek dapat diekstansi dari citra pada permulaan pra-pengolahan dan segmentasi citra. salah satu tantangan utama pada komputer vision adalah merepresentasikan bentuk, atau aspek-aspek penting dari bentuk.
Untuk ukuran format SHP dan file komponen DBF tidak boleh melebihi dari 2 Gb (Gigabyte), dengan jumlah maksimum verteks atau titik (points) yaitu 70 juta fitur titik/vertex yang terbaik, sedangkan untuk jumlah maksimum dalam bentuk garis (lines) dan area (polygon) yang dapat ditampung tergantung dari jumlah verteks penyusun garis atau area yang digunakan. 
Kurang mendukung untuk nama field Unicode atau tempat penyimpanan field, panjang maksimum nama field adalah 10 karakter, dan jumlah maksimum dari field adalah 255.

\subsection{Struktur Data Shapefile}
Geodatabase adalah struktur data yang kuat dan canggih. selain topologi yang anda dapatkan secara gratis, area dan sekeliling area yang
digambarkan dengan fitur linier. Namun, ESRI juga mendukung struktur data yang jauh lebih rumit: shapefile. sebuah shapefile dapat menggabungkan dua elemen penting yang dimiliki oleh geodatabase(komponen geografis dan database atribut) Perangkat lunak 
basis data adalah sistem manajemen basis data yang dinamai dBASE. Shapefile tertentu dibatasi hanya untuk mewakili satu dari jenis berikut: titik, multipoint, polyLines, atau poligon dengan titik. masing-masing titik memiliki catatan database relasional. Jika sejumlah titik dianggap objek yang sama, maka objek tersebut hanya memiliki satu record di tabel atribut. seperti pada geodatabases, polylines dapat disusun dari satu atau lebih jalur terhubung ataupun terputus-putus. Namun, jalur diperbolehkan untuk disusun hanya dari segmen garis lurus. Poligon dalam shapefile memiliki kemiripan dengan poligon basis geodata, namun tidak ada topologi yang ada dan tidak ada yang dapat diciptakan. setiap poligon berdiri sendiri. Hal ini digambarkan secara lengkap oleh satu entitas linier: urutan segmen yang dimulai di satu lokasi geografis dan kembali ke lokasi tersebut. mungkin ada poligon yang berdekatan atau tidak. poligon lain mungkin tumpang tindih.

\subsection{Daftar beberapa file extension}
\begin{enumerate}
    \item 1.shp - File yang menyimpan feature geometri (diperlukan dalam sebuah shapefile) 
    \item 2.shx - File yang menyimpan index dari feature geometri (diperlukan dalam sebuah shapefile) 
    \item 3.dbf - File dBASE yang menyimpan informasi atribut dari suatu feature (diperlukan dalam sebuah shapefile) 
    \item 4.sbn dan *.sbx – File yang menyimpan spatial index dari feature (optional) 
    \item 5.fbn dan *.fbx – File yang menyimpan spatial index dari feature shapefile yang read-only (optional) 
    \item 6.ain dan *.aih – File yang menyimpan index atribut dari field yang aktif dalam sebuah tabel (optional) 
    \item 7.prj - File yang menyimpan informasi koordinat dari sebuah shapefile, file ini dapat muncul jika kita menggunakan ArcView Projection Utility (optional).
\end{enumerate}
\subsubsection{Contoh penggunaan extension shapefile}
\begin{enumerate}
    \item 1. read.shapefile (form.name) 
    \item 2. read.shp (shp.name) 
    \item 3. read.shx (shx.name)
    \item 4. read.dbf (dbf.name, sundulan = FALSE)
    \item 5. write.shapefile (shapefile, out.name, arcgis = FALSE) 
    \item 6. write.shp (shp, out.name) 
    \item 7. write.shx (SHX, out.name)
    \item 8. write.dbf (DBF, out.name, arcgis = FALSE) 
    \item 9. calc.header (shapefile) 
    \item 10. add.xy (shapefile)
\end{enumerate}
\subsubsubsection{Bentuk file}
\begin{itemize}
    \item *.scaleXY (shapefile, scale.factor)
    \item *.convert.to.shapefile (shpTable, attTable, bidang, jenis) 
    \item *.convert.to.simple (shp)
    \item *.change.id (shpTable, newFieldAsVector) 
    \item *.dp (poin, toleransi)
\end{itemize}

\subsection{Komponen Teknis}
Komponen yang ada pada sebuah aplikasi GIS
mempunyai fungsi utama untuk membaca dan menulis
data spasial, baik yang tersimpan dalam sebuah
shapefile (*.shp) atau tersimpan ke dalam sebuah
database (Eddy 2006).
Dalam MapServer yang sudah berjalan ada beberapa
Komponen utama yang digunakan secara peneuh untuk
menjalankan Aplikasi GIS untuk menangani data
spasial baik yang tersimpan dalam sebuah flat file atau
juga dalam DBMS yaitu :
1. SHAPELIB
Shapelib merupakan library yang ditulis
menggunakan bahasa pemrograman C yang
digunakan untuk melakukan proses read terhadap
Shapefile (*.shp) yang sudah didefinisikan ESRI
(Environmental System Research Institute).
Format dalam shapefile umum digunakan untuk
menyimpan data vector simple (tanpa topologi)
dengan atribut, shapefile merupakan format data
default yang digunakan dalam GIS.
