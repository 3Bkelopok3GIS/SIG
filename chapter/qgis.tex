%qgis
%kelompok 4 D4 TI-2D
%Ayu Permata Sari        1154022
%Librantara Erlangga     1154071
%Martin Luter Zega       1154120
%Putri Aulia Ramadhanie  1154096
%Ryan Hafizh Herdiana    1154067
%Copyright (c) 2017 Copyright Holder All Rights Reserved.

\section{sejarah qgis}
  qgis(dulu dikenal dengan nama quantum GIS) adalah salah satu perangkat lunak open source untuk keperulan GIS(Geographic information systems) yang paling populer, dengan basis penggunanya yang terus meningkat dan berkembang.
  aplikasi ini dapat melakukan banyak hal yaitu pembuatan data spasial,editing, dan juga analisis pemetaan.Selain berbasis dekstop aplikasi ini juga menyediakan bisa digunakan dalam web dan juga pada hp.

Quantum GIS dikembangkan dengan bahasa pemrograman C++ dan bersifat multi platform. Salah satu perangkat lunak ini dapat dijalankan pada berbagai sistem operasi. Saat ini, versi binary (installer) Quantum GIS tersedia untuk sistem operasi Microsoft Windows, Linux (berbagai varian distro), FreeBSD dan MacOS X. Quantum GIS bahkan sudah mulai dicoba dijalankan di sistem operasi Android yang banyak digunakan di perangkat mobile (smartphone/tablet). Saat ini versi stabil Quantum GIS adalah 1.8.0, dan sedang dalam tahap pengembangan untuk mencapai versi mayor 2.0.

Quantum GIS dapat digunakan untuk pengolahan data atribut secara umum seperti melakukan overlay layer, menguhitung luasan suatu wilayah, memberikan informasi tambahan pada suatu titik, ataupun merancang layout peta. Quantum GIS juga mendukung penggunaan GPS sehingga pengguna dapat langsung memindahkan data langsung dari GPS ke PC atau sebaliknya. Quantum GIS sangat layak untuk dijadikan alternative perangkat lunak pemetaan dalam berbagai keperluan.

 \subsection{latar belakang}
 qgis merupakan perangkat lunak berbasis open source yang merupakan projek dari GIS,yang dimulai pada tahun 2002. Tujuan awal proyek ini adalah untuk membuat spasial data. hari ini qgis sudah mencapai titik dimana qgis berevolusi di mana qgis digunakan oleh berbagai pengguna setiap hari untuk melihat data spasial,mengedit juga menganalisis , terutama dalam kegiatan pembelajaran.

Quantum GIS terdiri dari beberapa modul, yaitu :
1.	QGIS Desktop : untuk input data, penampilan data, query data, analisa data, dan presentasi dalam bentuk peta.
2.	QGIS Browser : untuk manajemen data (copy, paste, preview, delete).
3.	QGIS Server : untuk mengelola basis data special dan layanan peta yang mendukung koneksi web site.
4.	QGIS Client : framework aplikasi GIS berbasis web yang mengkonsumsi data dari Quantum GIS Server.

\subsection{Fungsi-Fungsi}
\subsubsection {Melihat Data}
Anda dapat melihat dan overlay data vektor dan raster dalam format dan proyeksi yang berbeda tanpa konversi ke format internal maupun umum. Format yang didukung termasuk:

Tabel spasial-enabled dan tampilan menggunakan PostGIS, SpatiaLite dan MSSQL Spasial, Oracle Spasial, format vektor yang didukung oleh perpustakaan OGR, termasuk ESRI shapefile, MapInfo, SDTS, GML dan banyak lagi, lihat bagian Pekerjaan dengan Data Vektor.

Format raster dan citra yang didukung dengan terpasangnya GDAL (Geospatial Data Abstraction Library) perpustakaan, seperti GeoTiff, ERDAS IMG, ArcInfo ASCII GRID, JPEG, PNG dan banyak lagi, lihat bagian Pekerjaan dengan Data Raster.

Data raster dan vektor GRASS dari basis data GRASS (lokasi/mapset). Lihat bagian GRASS GIS Integration.

Data spasial dalam jaringan sebagai Layanan OGC Web, termasuk WMS, WMTS, WCS, WFS, dan WFS-T. Lihat bagian Pekerjaan dengan Data OGC.

Pengenalan tools:
1.	Menu Bar, menyediakan berbagai macam fitur/fungsi pada Quantum GIS yang akan digunakan untuk pengoperasian lebih lanjut. Sebagaian besar fungsi yang ada pada menu bar juga terdapat pada tool bar.
2.	Tool bar, terdiri dari deretan gambar(icon) yang menyediakan akses cepat (shortcut_ pada fungsi-fungsi yang sebagaian besar sama seperti yang ada pada menu bar.
