%OPENLAYER
%Kelompok 1 D4 TI 3D
%Wahyu Maruti Adjie_1154034
%Muhammad Nur Ikhsan_1154087
%Emy Safitri_1154102
%Andi Ikram Maulana_1154065
%Ilman Mubarik Sidiq_1154114

\documentclass{article}
\title{OPENLAYER}
\begin{document}
\maketitle

\section{deskripsi openlayer}
OpenLayers adalah library Javascript murni yang digunakan untuk menampilkan data peta di berbagai web browser, tanpa server side dependencies.
OpenLayer mengimplementasikan JavaScript API untuk membangun sebuah rich web-based geographic apllications yang menyerupai Google maps dan MSN Virtual Earth APIS.
Open Layer adalah Software, yang dibangun oleh komunitas Open Source.
  
OpenLayers Syntax
OpenLayers “Classes”
Ditulis menggunakan gaya klasik yang fungsi-fungsinya ditulis dengan dengan huruf besar pada awal syntax.
contohnya : var map = new OpenLayers.Map(“map”, options);
Syntax diatas menjelaskan sebuah map dengan semua property dari Open Layers. 
  
\subsection{Base Layers}
 Base Layers adalah layer yang bersifat mutually Exclusive yang berarti hanya satu yang dapat diaktifkan setiap saat.
 Base layer yang aktif akan menentukan proyeksi sistem koordinat yang ada dan memperbesar level yang ada dalam peta. 
 Base Layer atau bukan, hal ini ditentukan oleh property isBaseLayer pada layer. Kebanyakan Layer raster memiliki Properti isBaseLayer yang diset dengan true secara default. 
 Ini bisa diganti pada layer option. Base layers selalu ditampilkan dibawah overlay layers.

\subsection{Non Base Layers}
Non base layers, terkadang disebut juga sebagaioverlay, adalah merupakan Base Layers alternative. Non-base layer yang banyak dapat diaktifkan setiap saat.
Layer ini tidak memegang control untuk zoom level dari peta, tapi dapat diaktifkan atau di nonaktifkan pada skala yang tepat dengan min/max scale/parameter resolusi, sehingga hanya dapat diaktifkan pada level tertentu.

\subsection {Raster Layers}
Raster layer adalah layer bayangan. Layer ini merupakan fixed projection yang tidak dapat dirubah pada client side.

\subsection {Google}
Layer yang digunakan untuk data Google Maps PAA OpenLayer. Untuk informasi API, dapat dilihan pada Google Layer Api Docs. 
Contoh penggunaanya dapat dilihat pada contoh penggunaan Spherical Mercator.
Jika kita ingin untuk menumpang tindihkan (overlay) layer dasar Google Maps dengan data lain, kita harus berinteraksi dengan layer Google Maps pada koordinat proyeksi.
Kita dapat baca selengkapnya pada proyeksi Spherical Mercator dan berbagai layer komersial yang dipergunakan di dalam dokumentasi Spherical Mercator.
Class Google Layer di design hanya sebagai base layer.

\textbf{Image} \\
Sebagai informasi API selengkapnya lihat pada Image Layer API Docs.\\
\textbf{KaMap} \\
Sebagai informasi API selengkapnya lihat pada KaMap Layer API Docs.\\
\textbf{KaMapCache} \\
Sebagai informasi API selengkapnya lihat pada KaMapCache Layer API Docs.\\
\textbf{MapGuide} \\
Sebagai informasi API selengkapnya lihat pada MapGuide Layer API Docs.\\
\textbf{MapServer} \\
Ini tidak diperlukan untuk berinteraksi dengan Map Server. Secara umum, lebih disukai daripada Layer MapServer, karena MapServer mengekspos sebagian besar fungsi CGI dala modus WMS.
Lapisan MapServer seing dapat menyebankan peta yang tampaknya bekerja, tetapi masih bermasalah pada proyeksi atau misconfigurasi seruma lainnya.
Dalam menggunakan WMS kita harus memilika alas an yang kuat dibandingkan dengan menggunakan MapServer.
Jika kita menggunakan layer MapServer, dan peta sedang diulang beberapa kali, ini menunjukkan bahwa kita belum melakukan konfigurasi dengan peda dan masih dalam proyeksi yang berbda.
openLayer tidak dapat membaca informasi dari MAPFILE, dan harus dikonfigurasi secara explicit.\\

\subsection {Overlay layer}
Merupakan lapisan yang memiliki sumberdata dalam format selain citra. Ini termasuk subclass dari kedua layer Marker dan layer Vector.

\subsection {Boxes}
boxes lebih baik untuk mengimplementasikan fungsi dengan vector layer dan polygon geometric dengan mempertahankan untuk kompatibilitas mundur.

\subsection {GML}
GML layer merupakam subclass dari vector layer yang berguna untuk membaca data dan menampilkannya. 
Contoh :
var layer = new OpenLayers.Layer.GML(“GML”, “gml/polygon.xml”)
map.addLayer(layer);

\end{document}