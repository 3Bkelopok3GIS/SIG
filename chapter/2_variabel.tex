\section{Variabel}
Pada sebagian besar bahasa pemrograman, nama suatu variabel
menjelaskan suatu nilai dengan tipe data tertentu 
dan menempati alamat memory yang pasti.
Variabel menyimpan data atau nilai yang dilakukan selama program dieksekusi,
Nilai variabel tersebut dapat diganti-ganti, namun tipe data selalu tetap.
Tidak demikian dengan python dimana tipe datanya dapat diubah-ubah
secara dinamis\cite{suparno2013komputasi}.

Variabel adalah entitas yang memiliki nilai dan berbeda satu dengan yang lain. Variabel mengalokasikan memori untuk menyimpan nilai.
Hal ini berarti ketika anda membuat variabel maka anda memesan beberapa ruang di memori. 
Variabel bisa digunakan untuk menyimpan bilangan bulat, desimal juga karakter.
Python sangat mementingkan indentasi, sehingga kita perlu melakukan indentasi secara konsisten. 
Indentasi tersebut dipermudah dengan penggunaan tombol Tab dan dimulai dari kolom pertama untuk setiap blok baru. \cite{santoso2009bahasa}


Variabel pada Python memiliki beberapa aturan seperti :
•	Case Sensitive : penggunaan huruf besar dan huruf kecil yang dibedakan.
•	Harus dimulai dengan underscore (_) atau huruf biasa, setelah itu dapat diikuti dengan huruf, angka atau underscore (_).
•	Tidak boleh mengandung karakter special seperti !,@,\#,\$ dan lainnya.
•	Hanya dapat menggunakan suatu variable setelah kita memberikan nilai ke dalamnya atau telah dilakukan assignment.
•	Setiap variable akan menyimpan referensi ke suatu objek dalam memory.\cite{santoso2009bahasa}
