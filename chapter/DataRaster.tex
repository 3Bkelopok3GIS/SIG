% Kelompok 2 Tugas 2 GIS (DATA RASTER)
% Tiara Rizki Wulansari (1154026)
% Muhamad Rifan Zamaludin (1154088)
% Mohammad Agung Deomartha (1154032)

\section{Data Raster}
\subsection{Pengertian Data Raster}
Data raster adalah data yang disimpan dalam bentuk kotak segi empat (grid) sel sehingga terbentuk suatu ruang yang 
teratur. Foto digital seperti areal fotografi atau satelit merupakan bagian dari data raster pada peta. 
Raster memiliki data grid continue. Nilainya menggunakan gambar berwarna seperti fotografi, yang ditampilkan dengan 
level merah, hijau, dan biru pada sel. Data Raster (atau disebut juga dengan sel grid) merupakan data yang dihasilkan dari sistem penginderaan jauh. Pada data raster. Obyek geografis direpresentasikan sebagai struktur sel grid yang disebut dengan pixel (picture element). Pada data raster. Resolusi (definisi visual) tergantung pada ukuran pixelnya. Dengan kata lain. Resolusi pixel menggambarkan ukuran sebenarnya dipermukaan bumi yang diwakili oleh setiap pixel pada citra.
\section{Data Raster}
\subsection{Akses ikonik ke repositori format data raster data monokrom elektronik jarak jauh}
Dalam Akses ikonik ke repositori format data raster data monokrom elektronik jarak jauh, 
Dokumen disimpan dalam sistem menggunakan monokrom, format raster. 
Dokumen dikirimkan dari repositori ke situs akses jarak jauh untuk ditampilkan kepada pengguna. 
Kemampuan tambahan disediakan untuk mencari dokumen yang tersimpan; 
menghasilkan layar antarmuka pengguna sesuai permintaan yang berisi hasil pencarian; 
memasukkan dokumen ke dalam repositori via transmisi oleh mesin faksimili; 
dan untuk berkomunikasi secara interaktif antara pengguna sistem. 
Dokumen elektronik bisa berupa teks dan grafis konvensional; 
atau dokumen multi media yang berisi teks, video, dan materi audio. 
Sebuah repositori dokumen fisik tunggal dapat secara logis tersegmentasi menjadi beberapa repositori virtual yang mendukung beragam kelompok pengguna.
