%OPENLAYER
%Kelompok 1 D4 TI 3D
%Wahyu Maruti Adjie_1154034
%Muhammad Nur Ikhsan_1154087
%Emy Safitri_1154102
%Andi Ikram Maulana_1154065
%Ilman Mubarik Sidiq_1154114

\section{deskripsi openlayer}
  OpenLayers adalah library Javascript murni yang digunakan untuk menampilkan data peta di berbagai web browser, tanpa server side dependencies.
  OpenLayer mengimplementasikan JavaScript API untuk membangun rich web-based geographic apllications yang menyerupai Google maps dan MSN Virtual Earth APIS.
  Open Layer adalah Software, yang dibangun oleh komunitas Open Source.
  
  OpenLayers Syntax
  OpenLayers “Classes”
  Ditulis menggunakan gaya klasik yang fungsi-fungsinya ditulis dengan dengan huruf besar pada awal syntax.
  contoh : var map = new OpenLayers.Map(“map”, options);
  Syntax diatas menjelaskan sebuah map dengan semua property dari Open Layers. 
  
\subsection{Base Layers}
 Base Layers adalah layer yang bersifat mutually Exclusive yang berarti hanya satu yang dapat diaktifkan setiap saat. Base layer yang aktif akan menentukan proyeksi sistem koordinat yang ada dan memperbesar level yang ada dalam peta. Base Layer atau bukan, hal ini ditentukan oleh property isBaseLayer pada layer. Kebanyakan Layer raster memiliki Properti isBaseLayer yang diset dengan true secara default. Ini bisa diganti pada layer option. Base layers selalu ditampilkan dibawah overlay layers.

\subsection{Non Base Layers}
Non base layers, terkadang disebut overlay, adalah merupakan Base Layers alternative. Non-base layer yang banyak dapat diaktifkan setiap saat. Layer ini tidak memegang control untuk zoom level dari peta, tetapi dapat diaktifkan atau di nonaktifkan pada skala yang tepat dengan min/max scale/parameter resolusi, sehingga hanya dapat diaktifkan pada level tertentu.


